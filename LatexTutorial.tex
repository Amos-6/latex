\documentclass[UTF8,a4paper]{ctexart}

%以下是导言区
\usepackage{geometry}
\geometry{left=2.5cm,right=2.0cm,top=2.5cm,bottom=2.0cm}
\usepackage{listings}
%以上是导言区

\begin{document}
\title{LaTex教程}
\author{amos}
\date{\today}
\maketitle

\begin{abstract}
    本教程按文档的格式要素来说明latex语法。
\end{abstract}
    
\section{标题与目录}
    \subsection{概述}
        本节的主要内容有:
            \begin{itemize}
                \item 文档的结构化
                \item 标签与交叉引用
                \item 设置标题样式
            \end{itemize}
        在book和report文档类中,可以使用\textbackslash part、\textbackslash chaper、\textbackslash section、\textbackslash subsection、\textbackslash subsubsection、\textbackslash paragraph、\textbackslash subparagraph这些章节命令,在\textbackslash article文档类中,除了\textbackslash chapter不能用,其他都可以用。

        用\textbackslash tableofcontents命令可以自动从各章节标题生成目录。在导言区中用下面的命令载入hyperref宏包\textbackslash usepackage\{hyperref\}就可以让生成的目录有链接,点击时会自动跳转到该章节。而且也会使得生成的pdf文件带有目录书签。

    \subsection{文档的结构化}
        在ctexart/article文类中,Latex提供了如下的小标题级别和对应的命令:

            \begin{enumerate}
                \item section(\textbackslash section\{节标题\})
                \item subsection(\textbackslash subsection\{小节标题\})
                \item subsubsection(\textbackslash subsection\{小小节标题\})
                \item paragraph(\textbackslash paragraph\{段落标题\})
                \item subparagraph(\textbackslash subparagraph\{小段落标题\})
            \end{enumerate}
    
        前三个级别是有编号的,后两个则没有编号。效果大概是这样的:

            \begin{tabular}{|c|}
                \hline
                \parbox{\textwidth}{
                    \section*{1.节}
                        这里一般是这一节的概述
                        \subsection*{1.1小节}
                            这里是第一小节的内容
                            \subsubsection*{1.1.1小小节}
                                第一小节中的第一点
                                \paragraph{段落}
                                    这是一个段落
                                    \subparagraph{小段落}
                                        这是一个小段落
                    \section*{2.又一节}
                        第二节概述
                        \subsection*{2.1又一小节}
                            第二节的第一小节
                }\\
                \hline
            \end{tabular}{|c|}

        在编译过程中,引擎会自动为这个节标题进行编号。这是一个非常好用的特性。如果你的文章有十个section,你想在第二个和第三个之间加入一个setion,只需要直接往里面加就行了,完全不用考虑编号的事。再比如有几个section你想要加深一级变成subsection,再合并成一个section,在LaTex这种组织结构下可以很轻松地完成。

        在编制目录的时候,我们有时候不希望给某一节编号,也不想它进入目录。这个时候只需要在命令后加上一个星号(*)就可以了,比如\textbackslash section*\{不编目的小节\}。

        在ctexart/report和ctexbook/book文类中,会有更高的节级别,比如part、chapter之类。chapter在学术论文的写作中还算常用,比section高一个级别,如果篇幅比较大的话可以更有效的组织文档。写作学术论文的时候一般会有模板可以使用,这时只要按照模板的说明操作就可以了。

    \subsection{标签与交叉引用}
        在行文过程中,我们可能需要引用自己前面写过的结果。比如“在第一章第5节我们提到......”“请参见我在第327页做的推导......”之类的。在这种情况下,自动化的需求是非常高的。如果说我手动输入了这些数字,那么一旦文档结构有变动,产生的工作量将非常可观。这些东西你还不能不改,因为你不改的话完全就是错的,本来想引用第3节结果写的是“见第2节”。但这在LaTex中完全不是问题。我们可以用标签(label)和交叉引用(cross reference)来解决这个问题。

        标签与交叉引用的基本思路是这样的:首先把我需要引用的章节用\textbackslash label\{标签名称\}进行标记,然后再需要引用它的时候用\textbackslash ref\{标签名称\}进行引用。这样一来,引擎在进行编译的时候就需要编译两次:第一次生成所有章节的编号,第二次再把这些编号填充到相应的位置。所以如果使用了这个命令但只编译了一次,这些引用的地方都将显示为问号。这对所有交叉引用都是一样的。下面我们来看一个使用的例子:

            \begin{tabular}{|c|}
                \hline
                \parbox{\textwidth}{

                }\\
                \hline
            \end{tabular}

    \subsection{设置标题样式}
        标题样式的设置分为三个部分,标题之前、标题、标题之后。其中标题包括标签和标题文字。标签为标题文字之前的内容,包括标题序号。

        titlesec 宏包提供了\textbackslash titleformat命令用来设置各级标题的样式,调用形式如下:

            \begin{tabular}{|c|}
                \hline
                \parbox{\textwidth}{
                    \textbackslash titleformat\{<command>\}[<shape>]\{<format>\}\{<label>\}\{<sep>\}\{<before-code>\}\{<after-code>\}
                }\\
                \hline
            \end{tabular}
        
        \begin{itemize}
            \item <command> 为被定义的标题命令。如:\textbackslash part, \textbackslash chapter, \textbackslash section, \textbackslash subsection, \textbackslash subsubsection, \textbackslash paragraph 或者 \textbackslash subparagraph。
            \item <shape> 为标题形式。
                \begin{itemize}
                    \item  hang 是默认的标题形式(和标准的\textbackslash section 形式一样)。
                    \item display 将标签单独作为一段(和标准的 \textbackslash chapter 形式一样)。
                    \item runin 下方段落和标题同行(和标准的\textbackslash paragraph 形式一样)。
                    \item frame 和 display 模式一样,只不过内容用一个盒子包裹。
                    \item block 将整个标题排版在一个没有附加形式的块(段落)中。在居中标题和特殊排版(包括图片工具)时有用。
                    \item leftmargin,rightmargin,drop,wrap用来将标题垂直排版在左侧或右侧,可能会造成重叠。具体用法和注意事项请参考源手册。
                \end{itemize}
            \item <format> 指定一个用在整个标题(包括标题文字和标签)的格式。如我们正文的文字使用罗马字体族,整个标题使用无衬线字体族,可以在这里输入\textbackslash sffamily 进行切换。这个地方可以添加一些垂直元素(对于某些垂直形式[shape]的标题为水平元素),这些元素会出现在标题上方空白的下方。
            \item <label> 用来定义标签。如默认section的label为 \textbackslash thesection\textbackslash quad,你可以使用 \textbackslash thesection.\textbackslash quad 在标签和文字之间加上一个点。如果你的标题等级中不需要 label。你可以将这个设置为空,但不建议这么做。因为这并不会抑制目录及栏外标题中的 label。
            \item <sep> 定义标签和标题文字之间的水平距离。必须有一个值,不能为空(可以为 0pt)。在 display shape 中为垂直距离。在 frame shape 中为标题文字到框(frame)之间的距离。在带星号的标题命令中 <label> 和 <sep> 都会被忽略。如果你使用图片之类的元素,请将该参数设置为 0 pt。
            \item <before-code> 出现在标题文字之前的代码,标题文字会作为参数自动的传递给最后一个命令。然而,如果加载包的时候使用了 explicit 参数,必须显示的使用 \#1 来替代标题文字。这在标题文字出现在命令中间时十分有用。
            \item <after-code> 出现在标题文字之后的代码。用于排版的元素在 hang,block 和 display 形式时为垂直模式。在 runin 和 leftmargin 为水平模式。其他模式下被忽略。
        \end{itemize}
        
        下面是一个排版示例:

        \begin{tabular}{|c|}
            \hline
            \parbox{\textwidth}{
                \textbackslash titleformat\{\textbackslash section\}[hang]\{\textbackslash sffamily\textbackslash vbox\{\textbackslash titlerule\}\}\{\textbackslash centering\textbackslash zihao\{-3\}\textbackslash bfseries \textbackslash S\textbackslash \textbackslash thesection\textbackslash enspace\}\{0pt\}\{\textbackslash zihao\{-3\}\textbackslash bfseries\}[\textbackslash vbox\{\textbackslash titlerule \textbackslash vspace\{1pt\} \textbackslash titlerule\}]
            }\\
            \hline
        \end{tabular}
       



    
  %字体字号加粗斜体下划线,字体颜色,背景色
        
    \section{页面设置}
    \subsection{页面大小和页边距}
    默认的页面大小是A4纸,输入以下内容,如果不进行设置的话,latex采用默认的页面大小,也即A4纸。
    
    \begin{lstlisting}
        \documentclass[]{article}
        \begin{document}
            Latex
        \end{document}
    \end{lstlisting}

    \part{代码块}



    \textbackslash documentclass[11pt,twoside,a4paper]\{article\}

    这条命令指定 LaTeX 使用论文版式,11 磅大小的字体来排版此文档,并且得到适合打印在 A4 纸上的输出结果。
    \newline
\begin{tabular}{|c|c|}
    \hline
    参数&含义\\
    \hline
    article &排版科技期刊、短报告、程序文档、邀请函等。\\
    report&排版多章节的长报告、短篇的书籍、博士论文等。\\
    book&排版书籍。\\
    slides&\parbox{\textwidth}{排版幻灯片。其中使用了较大的 sans serif字体。\\也可以考虑使用 FoilTEX 来得到相同的效果。}\\
    \hline
\end{tabular}
\end{document}